\documentclass[semifinal]{cpecmu}

%% This is a sample document demonstrating how to use the CPECMU
%% project template. If you are having trouble, see "cpecmu.pdf" for
%% documentation.

\projectNo{S007-2/66}
\acadyear{2024}

\titleTH{การพัฒนาซอฟต์แวร์ไมโครเซอร์วิสเชิงพาณิชย์ ณ บริษัท เอสซีบี เทคเอกซ์}
\titleEN{Commercial Microservices Software Developement at SCB TechX}

\author{นายธนภัทร สมสิทธิ์}{Thanapat Somsit}{640610639}

\cpeadvisor{patiwet}
\cpecommittee{karn}
\cpecommittee{navadon}

%% Some possible packages to include:
\usepackage[final]{graphicx} % for including graphics

%% Add bookmarks and hyperlinks in the document.
\PassOptionsToPackage{hyphens}{url}
\usepackage[colorlinks=true,allcolors=Blue4,citecolor=red,linktoc=all]{hyperref}
\def\UrlLeft#1\UrlRight{$#1$}

%% Needed just by this example, but maybe not by most reports
\usepackage{afterpage} % for outputting
\usepackage{pdflscape} % for landscape figures and tables. 

\usepackage{float} % for table
\usepackage{multirow} % Add this package in the preamble

%% Some other useful packages. Look these up to find out how to use
%% them.
% \usepackage{natbib}    % for author-year citation styles
% \usepackage{txfonts}
% \usepackage{appendix}  % for appendices on a per-chapter basis
% \usepackage{xtab}      % for tables that go over multiple pages
% \usepackage{subfigure} % for subfigures within a figure
% \usepackage{pstricks,pdftricks} % for access to special PostScript and PDF commands
% \usepackage{nomencl}   % if you have a list of abbreviations

\usepackage{pgfgantt} % for gantt chart

\usepackage{svg} % for svg

\usepackage{listings}
\usepackage{inconsolata}

\usepackage{pgf-pie}

% \usepackage{fontspec}
% \setmonofont{Consolas}

% \newfontfamily{\ttconsolas}{Consolas}

% \renewcommand{\ttdefault}{pcr}
% \newcommand{\listingsttfamily}{\fontfamily{inconsolata}\small}

\lstset{
  frame=tb,
  % language=GraphQL,
  aboveskip=3mm,
  belowskip=3mm,
  showstringspaces=false,
  columns=flexible,
  basicstyle={\small\ttfamily},
  numbers=none,
  breaklines=true,
  breakatwhitespace=true,
  tabsize=2
}

%% if you're having problems with overfull boxes, you may need to increase
%% the tolerance to 9999
% \tolerance=9999

% \bibliographystyle{plain}
% \bibliographystyle{IEEEbib}
\bibliographystyle{IEEEtran}

% \renewcommand{\topfraction}{0.85}
% \renewcommand{\textfraction}{0.1}
% \renewcommand{\floatpagefraction}{0.75}

%% Example for glossary entry
%% Need to use glossary option
%% See glossaries package for complete documentation.
\ifglossary
  \newglossaryentry{lorem ipsum}{
    name=lorem ipsum,
    description={derived from Latin dolorem ipsum, translated as ``pain itself''}
  }
\fi

%% Uncomment this command to preview only specified LaTeX file(s)
%% imported with \include command below.
%% Any other file imported via \include but not specified here will not
%% be previewed.
%% Useful if your report is large, as you might not want to build
%% the entire file when editing a certain part of your report.
% \includeonly{chapters/intro,chapters/background}

\begin{document}
\maketitle
\makesignature

\ifproject
\begin{abstractTH}
    ในรายงานนี้ ข้าพเจ้าได้สรุปประสบการณ์การปฏิบัติงานสหกิจศึกษา ณ บริษัท เอสซีบี เทคเอกซ์ จำกัด ซึ่งเป็นส่วนหนึ่งของหลักสูตรการศึกษาในระดับปริญญาตรี สาขาวิชาคอมพิวเตอร์ คณะวิศวกรรมศาสตร์ มหาวิทยาลัยเชียงใหม่ การปฏิบัติงานมุ่งเน้นไปที่การพัฒนาซอฟต์แวร์ในโครงการ xPlatform และการสนับสนุนงานในทีม Fast Easy ในฐานะวิศวกรซอฟต์แวร์
    
    นอกจากนี้ ข้าพเจ้ายังได้กล่าวถึงรางขอบเขตงานกระบวนวิชาสหกิจศึกษา (TOR) ซึ่งเป็นแนวทางสำคัญในการปฏิบัติงาน โดยมีวัตถุประสงค์เพื่อให้การฝึกงานเป็นไปตามมาตรฐานที่มหาวิทยาลัยกำหนด
    
    รายงานนี้จะนำเสนอรายละเอียดเกี่ยวกับประสบการณ์การทำงาน งานที่ได้รับมอบหมาย และผลลัพธ์ที่ได้จากการปฏิบัติงานในโครงการต่าง ๆ รวมถึงบทบาทของข้าพเจ้าในทีมพัฒนา ซึ่งจะช่วยให้ผู้ที่สนใจในสายงานนี้เข้าใจแนวทางและการดำเนินงานขององค์กรได้ดียิ่งขึ้น
\end{abstractTH}
    
    

% \begin{abstract}
%     The Screetner project (Street Scanner System for Urban Asset Management) is 
%     a project developed to facilitate the management of taxable billboards utilizing 
%     Object Detection technology. This is achieved through the use of a mobile application 
%     on handheld devices to capture image data, while simultaneously having a server to 
%     process the image data. Lastly, there is a web application to display reports derived 
%     from the captured data.
% \end{abstract}

\iffalse
\begin{dedication}
This document is dedicated to all Chiang Mai University students.

Dedication page is optional.
\end{dedication}
\fi % \iffalse

\begin{acknowledgments}
    การที่ข้าพเจ้าได้มาปฏิบัติงานสหกิจศึกษา ณ บริษัท เอสซีบี เทคเอกซ์ จำกัด ตั้งแต่วันที่ 4 เดือน มิถุนายน
     พ.ศ. 2567 ถึง วันที่ 25 เดือน ตุลาคม พ.ศ. 2567 ส่งผลให้ข้าพเจ้าได้รับ
    ความรู้และประสบการณ์ต่าง ๆ ที่มีค่ามากมาย สำหรับรายงานวิชาสหกิจศึกษาฉบับนี้ สำเร็จลงได้ด้วยดี
    จากความร่วมมือและสนับสนุนจากหลายฝ่าย ดังนี้
    \begin{enumerate}
        \item ...........................ตำแหน่ง........................
    \end{enumerate}
    และบุคคลท่านอื่น ๆ ที่ไม่ได้กล่าวนามทุกท่านที่ได้ให้คำแนะนำช่วยเหลือในการจัดทำรายงาน
    ข้าพเจ้าใคร่ขอขอบพระคุณผู้ที่มีส่วนเกี่ยวข้องทุกท่าน ที่มีส่วนร่วมในการให้ข้อมูล เป็นที่ปรึกษาใน
    การทำรายงานฉบับนี้จนเสร็จสมบูรณ์ ตลอดจนให้การดูแลและให้ความเข้าใจเกี่ยวกับชีวิตของการทำงาน
    จริง ข้าพเจ้าขอขอบคุณ ไว้ ณ ที่นี้

\acksign{2024}{10}{25}
\end{acknowledgments}%
\fi % \ifproject

\contentspage

\ifproject
\figurelistpage
\tablelistpage
\fi % \ifproject

% \abbrlist % this page is optional

% \symlist % this page is optional

% \preface % this section is optional


\pagestyle{empty}\cleardoublepage
\normalspacing \setcounter{page}{1} \pagenumbering{arabic} \pagestyle{cpecmu}

\chapter{\ifenglish Introduction\else บทนำ\fi}

\section{\ifenglish The Importance and Origin of the 
project\else ความสำคัญและที่มาของโครงงาน\fi}

% Okay there is actaully a tab down here vvv
การจัดเก็บภาษีถือเป็นหนึ่งในรายได้หลักของประเทศไม่ว่าจะเป็นภาษีทางตรง อย่างเช่น ภาษีทางตรง ภาษีรายได้บุคคลธรรมดา ซึ่งจะจัดเก็บได้
จากประชาชนผู้มีเงินได้ทั่วไป ภาษีเงินได้นิติบุคคลซึ่งเป็นภาษีที่จัดเก็บได้จากเงินได้ของบริษัทหรือห้างหุ้นส่วนนิติบุคคล และยังมีภาษีทางอ้อม เช่น 
ภาษีมูลค่าเพิ่ม ภาษีธุรกิจเฉพาะ ซึ่งเงินที่ได้จากการเก็บภาษีเหล่าล้วนนำไปให้รัฐบาลใช้ในการพัฒนาประเทศให้เจริญก้าวหน้า  

ภาษีป้ายก็เป็นส่วนหนึ่งของรายได้ท้องถิ่นที่สามารถจัดเก็บได้โดยองค์กรปกครองส่วนท้องถิ่น โดยที่ภาษีลักษณะนี้เมื่อจัดเก็บได้แล้ว ทางท้องถิ่น
ไม่จำเป็นต้องส่งคืนให้ทางรัฐ สามารถนำไปใช้จัดการบริหารพัฒนาภายในท้องถิ่นของตนเองได้ 
แต่ด้วยความสามารถในการจัดเก็บภาษีป้ายขององค์กรปกครองส่วนท้องถิ่นในแต่ละที่ ขึ้นอยู่กับปัจจัยหลาย ๆ อย่าง เช่น 
การที่ไม่สามารถรู้ได้ว่าป้ายที่สามารถจัดเก็บภาษีได้นั้นอยู่ที่ตำแหน่งใดในเขตปกครอง ซึ่งมีส่วนทำให้ประสิทธิภาพในการค้นหาป้ายภายในท้องถิ่นที่มีอยู่ทำได้อยู่จำกัด
และเป็นขั้นตอนที่ต้องใช้กำลังคนในการตรวจสอบเป็นอย่างมาก 
ดังนั้นจากปัญหาในจุดที่กล่าวมาทำให้เกิดโครงงานที่เป็นเครื่องมือที่ช่วยในการตรวจจับหาป้ายที่คาดว่าจะสามารถนำไปจัดเก็บภาษี 
และรายงานผลให้กับแต่ละองค์กรปกครองส่วนท้องถิ่นให้ไปจัดเก็บภาษีจากป้ายเหล่านี้

\section{\ifenglish The Objective of the Study\else วัตถุปรสงค์ของการศึกษา\fi}
\begin{enumerate}
    \item ได้ประยุกต์ใช้ทักษะการพัฒนาซอฟต์แวร์ที่ในศึกษาในมหาวิทยาลัยและแหล่งอื่น ๆ บนโครงงานที่อยู่ในอุตสาหกรรมซิฟต์แวร์จริง
    \item ได้ประสบการณ์ในการพัฒนาเว็บแอปพลิเคชันในอุตสาหกรรม
    \item ได้ซึกษาระบบการทำงานของการเป็นผู้พัฒนาซอฟต์แวร์ รวมไปถึงเทคโนโลยีต่าง ๆ ที่ช่วยในการพัฒนาซอฟต์แวร์
\end{enumerate}

% \include{chapters/background}
% \include{chapters/approach}

\chapter{\ifenglish General Company Information\else ข้อมูลทั่วไปของบริษัท\fi}

การทําโครงงานเริ่มต้นด้วยการศึกษาค้นคว้าทฤษฎีที่เกี่ยวข้อง หรืองานวิจัย/โครงงานที่เคยมีผู้พัฒนาและนําเสนอไว้แล้ว ซึ่งเนื้อหาในบทนี้ก็จะเกี่ยวกับ
การอธิบายถึงทฤษฎีที่นำไปประยุกต์ใช้กับโครงงานนี้ เพื่ออำนวยให้ผู้อ่านทำความเข้าใจกับตัวระบบของโครงงานได้ง่ายขึ้น

\section{\ifenglish Company Products and Services\else สินค้าและบริการของบริษัท \fi}
\begin{enumerate}
  \item lorem
  \item ipsum
  \item solor
\end{enumerate}


\section{สถานะงบการเงิน}
ipsum
\chapter{\ifenglish Assigned Work and Terms of Reference\else งานที่รับมอบหมายและรางขอบเขตงาน\fi}
please extend this


\section{\ifenglish Terms of Reference for Cooperative Education\else รางขอบเขตงานกระบวนวิชาสหกิจศึกษา \fi}
Starting salary: 30k baht. Increase once a year, based on the employee's performace.

\section{\ifenglish Assigned Work\else งานที่ได้รับมอบหมาย \fi}
% TODO: lengthen this
งานที่ได้รับมอบหมาย โดยส่วนใหญ่แล้วจะเป็นงานที่ได้ทำงานกับทีม xPlatform ซึ่งเป็นโปรเจคใหญ่ของบริษัท เอสซีบี เทคเอกซ์ ด้วยเช่นกัน โดยงานที่ได้รับมอบหมายจะสามารถแบ่งออกได้เป็น 4 งานหลักดังนี้

\subsection{\ifenglish xPlatform Change Runbook​ Feature\else ฟีเจอร์ xPlatfrom Change Runbook\fi}
% TODO: clarify this
ในขั้นตอนของการพัฒนาซอฟต์แวร์นั้น ถ้าหากว่าผู้พัฒนานั้นจำเป็นต้องการไปปรับเปลี่ยน configuration ของระบบต่าง ๆ ที่เกี่ยวข้องนั้น ผู้พัฒนาการจะไม่สิทธิในการที่จะไปปรับเปลี่ยนส่วนนั้นได้โดยตรง อย่างเช่น ขั้นตอนการ deploy แต่ละส่วนของระบบรวม การปรับเปลี่ยนสิทธิการเข้าถึงข้อมูล ซึ่งการเปลี่ยนแปลงเหล่านี้จะต้องไปแจ้งพนักงานในแผนกอื่น ๆ ที่มีสิทธิในการเข้าถึงเท่านั้นอย่างเช่น อย่างเช่น DevOps IT แผนกผู้บริหารฐานข้อมูล 

การทำงานแต่ละขั้นตอน จะมีชื่อเรียกว่า Activity รายงานขั้นตอนของการทำงานที่จะแจ้งแผนกต่าง ๆ นั้นจะมีชื่อว่า Runbook โดยที่ขั้นตอนดังกล่าวนี้โดยปกติจะทำรวมกับการเปลี่ยนแปลงเวอร์ชั่นของซอฟต์แวร์ที่จะเรียกว่า Change หรือที่มักจักเป็นที่รู้จักกันว่า Release โดยปกติแล้ว การขั้นตอนการเขียน Runbook นั้นจะลงเองด้วยมือ ซึ่งเป็นเรื่องที่ค่อนข้างเสียเวลามาก และสามารถเกิดข้อผิดพลาดขณะการเขียนได้ง่าย เราจึงได้สร้างฟีเจอร์ Change Runbook เพื่อช่วยให้นักพัฒนาซอฟต์แวร์สามารถรายงานขั้นตอนการทำงานได้สะดวกขึ้น

\[\text{add a blurred change runbook image here}\]

โดยที่ฟีเจอร์นี้จะมีความต้องการดังนี้
\begin{enumerate}
    \item ในแต่ละ Change จะมีอยู่หนึ่ง Runbook โดยที่ แต่ละ Runbook จะมีอยู่หลาย ๆ กลุ่มงาน (Activity Groups) แล้วแต่ละ Activity Groups จะมีอยู่หลาย ๆ Activities ในแต่ละ Activity จะต้องประกอบไปด้วยข้อมูล 
    \begin{enumerate}
        \item ชื่อ (Title)
        \item รายละเอียด (Description) 
        \item แท็ก (Hashtag)
        \item ผู้ที่รับผิดชอบ (Owner) (แผนกหรือพนังงานที่มีส่วนเกี่ยวข้องในการทำงาน)
        \item เวลาเริ่มต้นและเวลาสิ้นสุดของการทำงานขั้นตอนนั้น ๆ (Impl-start กับ Impl-end)
        \item Activities ที่จะต้องถูกทำงานเสร็จก่อน (Dependency)
        \item ประเภทของ Activity (Deploy กับ Rollback)
        \item สถานะการทำงาน (กำลังดำเนินอยู่ สำเร็จ ล่าช้า 10 นาที ล่าช้า 20 นาที และ ล่าช้าจนมีผลกระทบ)
        \item ความก้าวหน้าของงาน (0\% 20\% 40\% 60\% 80\% และ 100\%)
    \end{enumerate}
    ซึ่งจะมีแผนผังแสดงความสัมพันธ์ระว่าง Entity ดังนี้
    \[\text{insert ER diagram here}\]
    \item ผู้ที่จะสามารถเปลี่ยนแปลงข้อมูล (Update) หรือลบ (Delete) Activity ได้ จะเป็นผู้ที่สร้าง Activity นั้น ๆ หรือ Product Manager กับ Product Owner (PO \& PM)
    \item ในแต่ละ Activity จะสามารถเปลี่ยนแปลงสถานะการทำงานหรือความก้าวหน้าของงานได้ ซึ่งการทำเช่นนี้จะมีเรียกว่าการ Marking โดยที่ผู้ที่จะสามารถ Mark ได้จะเป็นเพียงแค่ผู้ที่มีหน้าที่รับผิดชอบ (Responsible people) หรือผู้ใช้ที่มีหน้าที่เป็น PO \& PM ซึ่งผู้ Mark จะสามารถระบุโน้ต หรือว่า Issue ที่เกี่ยวข้องกับการเปลี่ยนแปลงนั้นได้
    \item ในแต่ละ Activity จะสามารถแบ่งวิธีการหนดเวลาได้เป็น 2 รูปแบบ ได้แก่ Absolute กับ Relative โดยที่ 
    \begin{enumerate}
        \item Absolute Activity คือ Activity ที่ในขณะที่ถูก Create หรือ Update นั้น ผู้ใช้งานจะต้องระบุเวลาเริ่มต้นและเวลาจบของงาน โดยที่เวลาเริ่มต้นของ Activity ดังกล่าวต้องมาหลังเวลาจบของทุก ๆ Dependency (Time Constraint)
        \item Relative Activity คือ Activity ที่ในขณะที่ถูก Create หรือ Update นั้น ผู้ใช้จะระบุเพียงแค่ระยะการทำงานของ Activity นั้น ๆ โดยที่เวลาเริ่มต้นกับเวลาจบนั้นจะขึ้นอยู่กับ Time Constraint กล่าวคือ เวลาเริ่มต้นของ Activity นั้น ๆ จะเท่ากับ Time Constraint เสมอ ซึ่งหมายความว่าทุก ๆ Relative Activity จะจำเป็นต้องมีอย่างน้อย 1 Dependency
    \end{enumerate}
    \item ในการ Update Activity นั้น อาจเกิดกรณีทีี Activity นั้นเป็น Dependency ของ Activity ตัวอื่น ๆ ได้ ซึ่งเวลาการทำงานของ Activity ดังกล่าวนั้นจะจำเป็นต้องเปลี่ยนไปอัตโนมัติตามกฎดังนี้
    \begin{enumerate}
        \item หาก Time Constraint ของ Absolute Activity ถูกเลื่อนไปอยู่หลัง Activity นั้น เวลาในการทำงานของ Activity จะถูกเลื่อนตามไปอยู่หลัง Time Constraint โดยผู้ใช้สามารถเลือกที่จะ Bypass Absolute Activity เพื่อไม่ให้เวลาการทำงานของ Activity เปลี่ยนแปลง แต่จะทำให้ความเป็น Dependency ของ Activities ที่เสร็จหลังก่อนที่ Absolute Activity จะเริ่ม นั้นถูกยกเลิก 
        \item Relative Activity จะต้องเปลี่ยนเวลาใหม่ถ้าหาก Time Constraint เปลี่ยน
    \end{enumerate}
    \[\text{add update example here}\]
    \item ในการ Delete Activity นั้น ถ้าหากตัวท่ีกำลังถูกลบอยู่เป็น Dependency ตัวเดียวของ Required Activity  Activity นั้นจะถูกโปรโมทให้เป็น Absolute Activity แทน
    \[\text{add delete example here}\]
    \item ผู้ใช้สามารถดึงข้อมูล (Import) จากไฟล์ประเภท CSV ได้ % TODO: complete this
    \item ผู้ใช้สามารถดึงข้อมูลของ Issues จากเว็บไซต์ Jira ในการสร้าง​ Activity ได้ โดยที่ผู้ใช้งานจะสามารถคัดเลือกข้อมูล (Query) ได้อยู่สองวิธี
    \begin{enumerate}
        \item การ Query แบบ Basic: ผู้ใช้จะต้องระบุ โค้ดของโปรเจค Label ของ Issue และ ประเภทของ Issue
        \item การ Query ด้วย Jira Query Lanauge (JQL) ซึ่งเป็นภาษาที่ช่วยในการค้นหาข้อมูลใด ๆ ก็ตามภายในเว็บไซต์ของ Jira
    \end{enumerate}
    โดยที่วิธีการดึงข้อมูลนี้จะแตกต่างกันตาม Type ของ Field ที่กำลังถูกดึง นอกจากนี้ Activity ที่ถูกดึงมา จะสามารถลิงก์กลับไปบนหน้่าเว็บเพจของ Issue นั้น ๆ บน Jira ได้ด้วยเช่นกัน
    \[\text{example jira issue}\]
    \item การ Import จากแหล่งใดก็ตามจะได้ประเภทการกำหนดเวลาแบบ Absolute เสมอ เนื่องจากการ Import จะไม่สามารถระบุ Dependency ได้ ผู้ใช้จะสามารถเพิ่ม Dependency ด้วยการ Update ทีหลัง
    \item ผู้ใช้การจะสามารถส่งออกข้อมูล (Export) ของ Runbook ออกเป็นไฟล์ .xlsx ได้

\end{enumerate}

\subsection{\ifenglish Fast Easy Tasks\else งานร่วมกับทีม fast easy\fi}

\subsection{\ifenglish Keycloak user credentail data synchronisation\else งานการบันทึก credential ของผู้ใช้ลงซอฟต์แวร์ Keycloak\fi}

\subsection{TBA}

\chapter{\ifenglish Work Result\else ผลของการทำงาน\fi}

\section{สรุปผลการทำงานตามรางขอบเขตงาน}
ตลอดระยะเวลาการทำงานที่บริษัท เอสซีบี เทคเอกซ์ ข้าพเจ้าได้สะสม Story Points รวมทั้งหมด xx คะแนน ซึ่งมากกว่าข้อกำหนดขั้นต่ำที่ระบุในรางขอบเขตงานที่ 60 Story Points อย่างไรก็ตาม ข้าพเจ้าไม่ได้มีโอกาสทำโปรเจคให้กับทีม Fast Easy ตามที่คาดหวังไว้ เนื่องจากทั้งสองทีมเห็นตรงกันว่าการย้ายข้าพเจ้าไปทำงานในทีม Fast Easy อาจไม่เหมาะสมมากนัก เนื่องจากข้าพเจ้าและเพื่อนร่วมฝึกงานยังไม่มีประสบการณ์กับโค้ดเบสของทีมนั้นมากนัก ในขณะที่มีประสบการณ์ในการพัฒนาบนโปรเจค xPlatform มากพอที่จะเริ่มงานอื่น ๆ ได้เลย

อย่างไรก็ตาม ข้าพเจ้ายังคงได้ทำงานร่วมกับทีม Fast Easy ในระดับเล็กน้อย ส่วนงานที่เหลือจะเป็นงานย่อยขนาดกลางในทีม xPlatform ตามที่ได้ระบุไว้ในบทก่อนหน้า

\section{สัดส่วนการทำงาน}
Story Points ส่วนใหญ่จะอยู่ที่ฟีเจอร์ xPlatform Change Runbook ซึ่งมีคะแนนรวมถึง 41 Story Points เนื่องจากโครงการนี้เป็นโครงการใหญ่ตามที่กำหนดในรางขอบเขตงาน ในโครงการจะมีการแบ่งงานออกเป็นงานย่อย โดยเฉลี่ยแต่ละงานจะมีคะแนนประมาณ 3 Story Points ส่วนชิ้นงานขนาดปานกลางและขนาดเล็กที่เหลือจะสะสมคะแนนรวมกันได้ทั้งหมด xx Story Points\enskip รายละเอียดคะแนนความยากของแต่ละงานจะถูกบันทึกไว้ในภาคผนวก

\[Insert Pie Chart Here\]

\section{ช่วงระยะเวลาการทำงาน}ข้าพเจ้าได้เริ่มต้นทำงานที่ฟีเจอร์ xPlatform เป็นระยะเวลาประมาณ 3 เดือน โดยในช่วงเวลาดังกล่าว ข้าพเจ้าได้ทำงานร่วมกับทีม Fast Easy เป็นระยะเวลา 1 สัปดาห์หลังจากเสร็จสิ้นโปรเจกต์แรก แลในระยะเวลาที่เหลือข้าพเจ้าได้มีส่วนร่วมในการทำงานในโปรเจกต์ขนาดกลางกับทีม xPlatform 
\chapter{\ifenglish Work environment\else สภาพแวดล้อมของการทำงานในองค์กร\fi}

\section{\ifenglish Before Internship\else ก่อนที่จะปฏิบัติงานสหกิจศึกษา\fi}
ก่อนที่จะได้ปฏิบัติงานสหกิจศึกษานั้น ข้าพเจ้าได้ยื่นสมัครไป 3 บริษัท ได้แก่ บริษัท อะแวร์ คอร์ปอเรชั่น จำกัด ในตำแหน่ง Java Developer บริษัท เอสซีบี เทคเอกซ์ ในตำแหน่ง Software Engineer และบริษัท ที.ซี.ซี. เทคโนโลยี จำกัด ในตำแหน่ง Software Engineer 

โดยที่แต่ละบริษัทจะมีขั้นตอนการสมัครงานดังนี้
\begin{enumerate}
    \item บริษัท อะแวร์ คอร์ปอเรชั่น จำกัด: 
    \begin{enumerate}
        \item เริ่มต้นด้วยการส่งจดหมายสมัครงานทางอีเมลถึงฝ่ายบุคคล โดยในอีเมลต้องแนบ ใบรับรองผลการศึกษา (Transcript) ประวัติย่อ (Resume) และวิดีโอแนะนำตัวเอง
        \item ทำการสอบสัมภาษณ์ออนไลน์เกี่ยวกับการแก้ปัญหาเบื้องต้น และความรู้พื้นฐานเกี่ยวกับภาษา Java ใช้เวลาทั้งหมด 1 ชั่วโมง
        \item เข้าร่วมสัมภาษณ์สหกิจศึกษา โดยจะสอบถามเกี่ยวกับโปรเจคที่เคยทำ และให้อธิบายคำตอบจากการสอบครั้งก่อน ใช้เวลาในการสัมภาษณ์ประมาณ 1 ชั่วโมง
    \end{enumerate}
    \item บริษัท เอสซีบี เทคเอกซ์
    \begin{enumerate}
        \item ส่งแบบฟอร์ม Google Form ซึ่งประกอบไปด้วยข้อมูลส่วนตัวอย่างละเอียด เช่น ประวัติการศึกษา ผลการเรียน Resume และตำแหน่งที่สนใจในการทำงาน
        \item ทำการสอบออนไลน์เกี่ยวกับการแก้ไขปัญหาเบื้องต้นผ่านแพลตฟอร์ม HackerRank
        \item เข้าร่วมสัมภาษณ์สหกิจศึกษา ซึ่งในระหว่างการสัมภาษณ์นี้จะมีการสอบถามเกี่ยวกับโปรเจคที่เคยทำ รวมถึงการให้ผู้สมัครอธิบายคำตอบจากการสอบครั้งก่อน
    \end{enumerate}
\end{enumerate}
ทั้งบริษัท อะแวร์ คอร์ปอเรชั่น จำกัด และบริษัท เอสซีบี เทคเอกซ์ ได้ติดต่อข้าพเจ้าเพื่อเข้าฝึกงานอย่างเป็นทางการ ส่วนบริษัท ที.ซี.ซี. เทคโนโลยี จำกัด ได้ติดต่อมาในภายหลัง หลังจากที่ข้าพเจ้ายืนยันการฝึกปฏิบัติงานสหกิจศึกษาที่บริษัท เอสซีบี เทคเอกซ์ ทำให้ข้าพเจ้าไม่มีโอกาสในการสมัครงานกับบริษัทดังกล่าว
\begin{table}[H]
    \centering
    \begin{tabular}{c||c}
        \textbf{วันที่} & \textbf{เหตุการณ์} \\ \hline
        13 ม.ค. 2567 & ส่งจดหลายสมัครบริษัท อะแวร์ \\ \hline
        16 ม.ค. 2567 & บริษัท อะแวร์ ส่งจดหมายเชิญชวนทดสอบความรู้ออนไลน์ \\\hline
        30 ม.ค. 2567 & ทดสอบความรู้ออนไลน์ กับบริษัท อะแวร์ \\\hline
        2 ก.พ. 2567 & บริษัท เอสซีบี เทคเอกซ์ ส่งจดหมายเชิญชวนทดสอบความรู้ออนไลน์ \\ \hline
        2 ก.พ. 2567 & บริษัท อะแวร์ ส่งจดหมายเชิญสัมภาษณ์ \\ \hline
        9 ก.พ. 2567 & ทำสัมภาษณ์กับบริษัท อะแวร์ \\ \hline
        TODO& EXTEND THIs \\ \hline
        5 มี.ค. 2567& บริษัท เอสซีบี เทคเอกซ์ส่งจดหมายแสดงความยินดีเข้าโปรแกรมฝึกงาน \\
    \end{tabular}
    \caption{ตารางแสดงไทม์ไลน์การหางาน}
    \label{tab:job-finding-timeline}
\end{table}

\section{\ifenglish Salary and Benefits \else เงินเดือนและสวัสดิการ \fi}
พนักงานทุกคนจะได้รับเงินเดือนเริ่มต้นที่ 30,000 บาท โดยเงินเดือนจะปรับขึ้นทุกปีตามผลการปฏิบัติงาน สำหรับนักศึกษาฝึกงานหรือสหกิจศึกษา จะได้รับเบี้ยเลี้ยงวันละ 500 บาท คิดเป็นประมาณ 10,000 บาทต่อเดือน

นอกจากเงินเดือนและเบี้ยเลี้ยงแล้ว พนักงานยังได้รับสิทธิประโยชน์เพิ่มเติมดังนี้:
\begin{enumerate}
    \item พนักงานแต่ละคนจะได้รับ MacBook Pro สำหรับใช้งานตลอดระยะเวลาการทำงาน โดยต้องส่งคืนเมื่อสิ้นสุดการจ้างงาน
    \item อาหารเช้าฟรีทุกวันทำงาน
    \item ในแต่ละปีพนักงานสามารถลาพักร้อน 15 วัน ลาเดือนเกิด 1 วัน ลากิจ 5 วัน และลาป่วย 30 วัน โดยสามารถสะสมวันลาพักร้อนข้ามปีได้สูงสุดถึง 5 วัน นอกจากนี้พนักงานสามารถลาบวช 105 วัน ซึ่งไม่เปลี่ยนแปลงในแต่ละปี
    \item นักศึกษาฝึกงานจะได้รับประกันชีวิตจำนวน 5000 บาท ประกันอุบัติเหตุ 150000 บาท และความคุ้มครองค่ารักษาพยาบาล (AME) อีก 5000 บาท โดยทั้งหมดเป็นแบบประกันกลุ่ม
    \item พนักงานทั่วไปจะได้รับประกันชีวิตจำนวน 1000000 บาท ประกันอุบัติเหตุจำนวน 1000000 บาท และประกันสุขภาพตามแผนที่พนักงานเลือก
    \item บริการรถรับส่งพนักงานระหว่างสำนักงานและสถาที่ต่าง ๆ สองรอบต่อวัน:
    \begin{enumerate}
        \item รอบเช้า: 6:30 – 9:00 น.
        \item รอบเย็น: 17:20 – 20:00 น.
    \end{enumerate}
    จุดขึ้น-ลงรถมี 4 สถานที่:
    \begin{enumerate}
        \item อาคาร SCB Plaza West
        \item BTS หมอชิต (ทางออก 2)
        \item MRT สวนจตุจักร (ทางออก 4)
        \item Big C วงศ์สว่าง
    \end{enumerate}
\end{enumerate}


\section{\ifenglish Corporate Culture \else วัฒนธรรมองค์กร \fi} 
บริษัทใช้รูปแบบการทำงานแบบไฮบริด 8 ชั่วโมงต่อวัน 9:00 - 18.00 น. มีพักเบรกรับประทานอาหารเที่ยง 1 ชั่วโมง โดยพนักงานจะเข้ามาทำงานที่สำนักงาน 2 วันต่อสัปดาห์ และทำงานทางออนไลน์อีก 3 วัน ส่วนวันที่ต้องเข้ามาทำงานที่สำนักงานจะมีการกำหนดให้แต่ละทีมผลัดกันมา

\subsection{กิจกรรมและการอบรม}
ในบริษัทจะมีการจัดกิจกรรมบ้างอย่างเช่น
\begin{enumerate}
    \item 
\end{enumerate}
และสำรับนักศึกษาฝึกงนก็จะมีกิจกรรมที่ต้องเข้าร่วมแบบ TODO: filthis
\begin{enumerate}
    \item 
\end{enumerate}
และก็มีกิจกรรมอีกมากมาย (TODO: report that i didn't listed all)

\subsection{ความคิดเห็นส่วนตัว}

\section{ข้อเสนอจากบริษัท}
ทางบริษัทไม่ได้ให้ข้อเสนอการทำงานทันทีหลังจากที่นักศึกษาจบการศึกษาในมหาวิทยาลัย อย่างไรก็ตามจากการพูดคุยกับผู้ที่เกี่ยวข้อง มีการกล่าวว่าผู้ที่ผ่านการปฏิบัติงานสหกิจศึกษาหรือฝึกงานกับบริษัทมีโอกาสค่อนข้างสูงที่จะได้รับการพิจารณาในการรับเข้าทำงานหลังจากสำเร็จการศึกษา แม้ว่าจะยังไม่มีการรับรองอย่างเป็นทางการในเรื่องนี้



% \ifproject
% \include{chapters/conclusion}
% \fi

\bibliography{ScreetnerReport}

% \ifproject
% \normalspacing
% \appendix
% \newcommand{\includepdfwithfirstpagefit}[1]{
    \includegraphics[width=\textwidth, height=\textheight, keepaspectratio]{#1}
    \includepdf[pages=2-, fitpaper=true, nup=1x1, frame=false]{#1}
}

\newcommand{\includepdfwithonepage}[1]{
    \includegraphics[width=\textwidth, height=\textheight, keepaspectratio]{#1}
}

\chapter{เอกสารเกี่ยวข้อง}

TODO add 3 and 3-4

\section{วศ.สก.-06}
\includepdfwithfirstpagefit{resources/bureaucrats/coop6.pdf}

\section{วศ.สก.-10}
\includepdfwithfirstpagefit{resources/bureaucrats/coop10.pdf}

\section{วศ.สก.-11}
\includepdfwithonepage{resources/bureaucrats/coop11.pdf}

\section{หนังสือยินยอมให้เผยแพร่รายงานปฏิบัติงานสหกิจศึกษา}
\includepdfwithonepage{resources/bureaucrats/publish-consent.pdf}

\chapter{ตารางแสดงรายละเอียดการสะสม Story Points}
\section{ตารางแสดงรายละเอียดการสะสม Story Points ของ Change Runbook}
\begin{table}[H]
    \centering
    \begin{tabularx}{0.85\textwidth}{X|c}
        \attr{รายละเอียด} & \attr{Story Points} \\
        \hline\hline
        \textbf{Change Runbook} & \textbf{42} \\
        getActivity & 2 \\
        getActivityInfo & 1 \\
        searchActivity & 1 \\
        getRequiredActivity & 2 \\
        getUserActivityPermission &	1 \\
        getResponsibleUser & 1 \\
        getLatestActivityMark &	1 \\
        Initial database design & 1 \\
        Import Jira - getField & 3 \\
        Import Jira - getPreview & 1 \\
        Import Jira - activity & 5 \\
        Import CSV - getField & 1 \\
        Import CSV - getPreview & 1 \\
        Import CSV - activity & 2 \\
        Optimise getActivity with resolver facilitator & 2 \\
        Autogeneration on activity group and activity hashtag on create/update activity & 1 \\
        Autodeletion on activity group and activity hashtag on update/remove activity & 1 \\
        Improve on change runbook .xlsx export & 5 \\
        getLatestImplementationDateTimeTo & 1 \\
        Implement previewUpdateActivityAffect and previewRemoveActivityAffect & 5 \\
        Implement update and remove activity time shift & 2 \\
        Change update/remove activity validation & 2 \\
        \hline\hline
        \textbf{รวม} & 42
    \end{tabularx}
    \caption{ตารางแสดงรายละเอียดการสะสม Story Points ของ Change Runbook}
    \label{tab:story-point-table}
  \end{table}

  \section{ตารางแสดงรายละเอียดการสะสม Story Points อื่น ๆ}
  \begin{table}[H]
      \centering
      \begin{tabularx}{0.85\textwidth}{X|c}
          \attr{รายละเอียด} & \attr{Story Points} \\
          \hline\hline
          \textbf{User Management} & \textbf{1} \\
          TODO & 2351 \\
          TODO & 1954 \\
          \hline
          \textbf{Documentation} & \textbf{1} \\
          TODO & 2351 \\
          TODO & 1954 \\
          \hline
          \textbf{Custom Library} & \textbf{1} \\
          TODO & 2351 \\
          TODO & 1954 \\
          \hline
          \textbf{Others} & \textbf{1} \\
          TODO & 2351 \\
          TODO & 1954 \\
          \hline\hline
          \textbf{รวม} & TODO
      \end{tabularx}
      \caption{ตารางแสดงรายละเอียดการสะสม Story Points อื่น ๆ }
      \label{tab:story-point-table-others}
    \end{table}

\chapter{การใช้งานซอฟต์แวร์}

\section{การใช้งานฟีเจอร์ Change Runbook}
\subsection{การสร้าง Activity}
\subsection{การเปลี่ยนแปลง Activity}
\subsection{การลบ Activity}
\subsection{การ Mark Activity}
\subsection{การดึงข้อมูลจาก Jira}
\subsection{การดึงข้อมูลจาก CSV}
\subsection{การส่งออกไฟล์ Excel}

\newpage
\section{การใช้งานฟีเจอร์ Custom Library}
\subsection{การสร้าง Custom Library Repostory}
\begin{center}
    \includegraphics[width=\linewidth]{resources/pages/custom-library/create-library/1.png}

    \vspace{1in}

    \includegraphics[width=\linewidth]{resources/pages/custom-library/create-library/2.png}
\end{center}
\begin{center}
    \includegraphics[width=\linewidth]{resources/pages/custom-library/create-library/3.png}

    \vspace{1in}

    \includegraphics[width=\linewidth]{resources/pages/custom-library/create-library/4.png}
\end{center}

\begin{figure}[H]
    \begin{center}
        \includegraphics[width=\linewidth]{resources/pages/custom-library/create-library/5.png}
    
        \vspace{1in}
    
        \includegraphics[width=\linewidth]{resources/pages/custom-library/create-library/6.png}
    \end{center}
    \caption[การสร้าง Custom Library]{การสร้าง Custom Library}
  \label{fig:create-library}
\end{figure}

\newpage
\subsection{การอัพเดตเวอร์ชัน Custom Library}
\begin{center}
    \includegraphics[width=\linewidth]{resources/pages/custom-library/update-library/6.png}

    \vspace{1in}

    \includegraphics[width=\linewidth]{resources/pages/custom-library/update-library/7.png}
\end{center}

\begin{figure}[H]
    \begin{center}
        \includegraphics[width=\linewidth]{resources/pages/custom-library/update-library/8.png}
    
        \vspace{1in}
    
        \includegraphics[width=\linewidth]{resources/pages/custom-library/update-library/9.png}
    \end{center}
    \caption[การอัพเดตเวอร์ชัน Custom Library]{การอัพเดตเวอร์ชัน Custom Library}
  \label{fig:update-library}
\end{figure}


\newpage
\section{การใช้งานฟีเจอร์ Documentation}
\begin{center}
    \includegraphics[width=\linewidth]{resources/pages/documentation/1.png}

    \vspace{1in}

    \includegraphics[width=\linewidth]{resources/pages/documentation/2.png}
\end{center}

\begin{figure}[H]
    \begin{center}
        \includegraphics[width=\linewidth]{resources/pages/documentation/3.png}
    \end{center}
    \caption[การใช้งานฟีเจอร์ Documentation]{การใช้งานฟีเจอร์ Documentation}
  \label{fig:documentation}
\end{figure}

%% Display glossary (optional) -- need glossary option.
% \ifglossary\glossarypage\fi

%% Display index (optional) -- need idx option.
% \ifindex\indexpage\fi

% \begin{biosketch}
% \begin{center}
%   \includegraphics[width=1.5in]{mugshot.jpg}
% \end{center}
% Your biosketch goes here. Make sure it sits inside
% the \texttt{biosketch} environment.
% \end{biosketch}
% \fi % \ifproject
\end{document}
