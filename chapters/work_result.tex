\chapter{\ifenglish Work Result\else ผลของการทำงาน\fi}

\section{สรุปผลการทำงานตามรางขอบเขตงาน}
% TODO: check accumulative story points
ตลอดระยะเวลาการทำงานที่บริษัท เอสซีบี เทคเอกซ์ ข้าพเจ้าได้สะสม Story Points รวมทั้งหมด xx คะแนน ซึ่งมากกว่าข้อกำหนดขั้นต่ำที่ระบุในรางขอบเขตงานที่ 60 Story Points อย่างไรก็ตาม ข้าพเจ้าไม่ได้มีโอกาสทำโปรเจคให้กับทีม Fast Easy ตามที่คาดหวังไว้ เนื่องจากทั้งสองทีมเห็นตรงกันว่าการย้ายข้าพเจ้าไปทำงานในทีม Fast Easy อาจไม่เหมาะสมมากนัก เนื่องจากข้าพเจ้าและเพื่อนร่วมฝึกงานยังไม่มีประสบการณ์กับโค้ดเบสของทีมนั้น ในขณะที่มีประสบการณ์ในการพัฒนาบนโปรเจค xPlatform มากพอที่จะเริ่มงานอื่น ๆ ได้เลย

อย่างไรก็ตาม ข้าพเจ้ายังคงได้ทำงานร่วมกับทีม Fast Easy ในระดับเล็กน้อย ส่วนงานที่เหลือจะเป็นงานย่อยขนาดกลางในทีม xPlatform ตามที่ได้ระบุไว้ในบทก่อนหน้า

\section{สัดส่วนการทำงาน}
Story Points ส่วนใหญ่จะอยู่ที่ฟีเจอร์ xPlatform Change Runbook ซึ่งมีคะแนนรวมถึง 41 Story Points เนื่องจากโครงการนี้เป็นโครงการใหญ่ตามที่กำหนดในรางขอบเขตงาน ในโครงการจะมีการแบ่งงานออกเป็นงานย่อย โดยเฉลี่ยแต่ละงานจะมีคะแนนประมาณ 3 Story Points ส่วนชิ้นงานขนาดปานกลางและขนาดเล็กที่เหลือจะสะสมคะแนนรวมกันได้ทั้งหมด xx Story Points\enskip รายละเอียดคะแนนความยากของแต่ละงานจะถูกบันทึกไว้ในภาคผนวก

% TODO: recheck about this
\begin{table}[H]
    \centering
    \begin{tabular}{c||c|c}
        & \attr{Story Points} & \attr{อัตราส่วน} \\
        \hline\hline
        \attr{Change Runbook} & 41 & 41\% \\
        \attr{User Management} & 20 & 41\% \\
        \attr{Custom Library} & 20 & 41\% \\
        \attr{Documentation} & 20 & 41\% \\
        \attr{Database Configration} & 1 & 5\% \\
    \end{tabular}
    \caption{ตารางแสดงสัดส่วนของ Story Points}
    \label{tab:story-point-table}
\end{table}

\begin{figure} [H]
    \begin{center}
        \begin{tikzpicture}
            \pie[radius=2]{41/Change Runbook,
                20/User Management,
                20/Custom Library,
                18/Documentation,
                1/Database Configuration
                }
        \end{tikzpicture}
    \end{center}
    \caption[แผนภูมิรูปวงกลมแสดงสัดส่วนของ Story Points]{แผนภูมิรูปวงกลมแสดงสัดส่วนของ Story Points}
    \label{fig:story-point-pie-chart}
\end{figure}

\section{ช่วงระยะเวลาการทำงาน}ข้าพเจ้าได้เริ่มต้นทำงานที่ฟีเจอร์ xPlatform เป็นระยะเวลาประมาณ 3 เดือน โดยในช่วงเวลาดังกล่าว ข้าพเจ้าได้ทำงานร่วมกับทีม Fast Easy เป็นระยะเวลา 1 สัปดาห์หลังจากเสร็จสิ้นโปรเจกต์แรก และในระยะเวลาที่เหลือข้าพเจ้าได้มีส่วนร่วมในการทำงานในโปรเจกต์ขนาดกลางกับทีม xPlatform ซึ่งจะได้แผนปฏิบัติงานสหกิจศึกษาดังนี้

\newcommand{\gantttitlevertical}[1]{\gantttitle{\rotatebox{90}{#1}}{1}}
\newcommand{\gantttitles}[2]{%
  \foreach \x in {#1} {%
    \gantttitle{\x}{#2}%
  }%
}
\newcommand{\blackcell}{
    \ganttbar[bar/.append style={fill opacity=1, pattern=crosshatch, pattern color=black},progress=100,inline=false]{}{16}{16}}

% link: https://tex.stackexchange.com/a/579761
\begin{table}[H]
    \centering
    \begin{ganttchart}[
        y unit title=0.5cm,
        y unit chart=0.6cm,
        x unit =0.6cm,
        % vgrid={draw=lightgray,line width=0.2pt},
        % hgrid={draw=lightgray,line width=0.2pt},
        vgrid,
        hgrid,
        title height=1,
        title/.style={fill=none, draw=black, line width=0.2pt},
        title label font=\footnotesize,
        bar/.style={fill=gray, fill opacity=0.5},
        bar height=1,
        bar top shift=0,
        progress label text={},
        group right shift=0,
        group height=.6,
        group peaks width={0.2},
        inline, 
        bar label node/.style={text width=2.75cm,
                               align=right,
                               anchor=east,
                               font=\small}
       ]{1}{16}
   
    \gantttitle{2024}{16}\\
                 
    \gantttitle{ก.ค.}{4} \gantttitle{ส.ค.}{4} \gantttitle{ก.ย.}{4} \gantttitle{ต.ค.}{4}\\

    \gantttitles{1,2,3,4,1,2,3,4,1,2,3,4,1,2,3,4}{1}\\
    
    \ganttbar[progress=100,inline=false]
        {Change Runbook}{1}{9}
    \ganttbar[progress=100,inline=false]
        {}{11}{11}
        \\
    \ganttbar[progress=100,inline=false]
        {DB Configuration}{10}{10}
        \\
    \ganttbar[progress=100,inline=false]
        {Documentation}{12}{12}
    \ganttbar[progress=100,inline=false]
        {}{13}{13}
        \\ 
    \ganttbar[progress=100,inline=false]
        {Custom Library}{12}{14} 
        \\
    \ganttbar[progress=100,inline=false]
        {User Management}{12}{12}
    \ganttbar[progress=100,inline=false]
        {}{14}{16} 
        \\
     \ganttbar[progress=100,inline=false]
        {Others}{10}{11} 
    \ganttbar[progress=100,inline=false]
        {}{13}{14} 

    \end{ganttchart}
    \caption{ตารางแผนปฏิบัติงานสหกิจศึกษา}
    \label{tab:work-timeline}
\end{table}