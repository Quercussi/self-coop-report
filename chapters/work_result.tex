\chapter{\ifenglish Work Result\else ผลของการทำงาน\fi}

\section{สรุปผลการทำงานตามรางขอบเขตงาน}
ตลอดระยะเวลาการทำงานที่บริษัท เอสซีบี เทคเอกซ์ ข้าพเจ้าได้สะสม Story Points รวมทั้งหมด xx คะแนน ซึ่งมากกว่าข้อกำหนดขั้นต่ำที่ระบุในรางขอบเขตงานที่ 60 Story Points อย่างไรก็ตาม ข้าพเจ้าไม่ได้มีโอกาสทำโปรเจคให้กับทีม Fast Easy ตามที่คาดหวังไว้ เนื่องจากทั้งสองทีมเห็นตรงกันว่าการย้ายข้าพเจ้าไปทำงานในทีม Fast Easy อาจไม่เหมาะสมมากนัก เนื่องจากข้าพเจ้าและเพื่อนร่วมฝึกงานยังไม่มีประสบการณ์กับโค้ดเบสของทีมนั้นมากนัก ในขณะที่มีประสบการณ์ในการพัฒนาบนโปรเจค xPlatform มากพอที่จะเริ่มงานอื่น ๆ ได้เลย

อย่างไรก็ตาม ข้าพเจ้ายังคงได้ทำงานร่วมกับทีม Fast Easy ในระดับเล็กน้อย ส่วนงานที่เหลือจะเป็นงานย่อยขนาดกลางในทีม xPlatform ตามที่ได้ระบุไว้ในบทก่อนหน้า

\section{สัดส่วนการทำงาน}
Story Points ส่วนใหญ่จะอยู่ที่ฟีเจอร์ xPlatform Change Runbook ซึ่งมีคะแนนรวมถึง 41 Story Points เนื่องจากโครงการนี้เป็นโครงการใหญ่ตามที่กำหนดในรางขอบเขตงาน ในโครงการจะมีการแบ่งงานออกเป็นงานย่อย โดยเฉลี่ยแต่ละงานจะมีคะแนนประมาณ 3 Story Points ส่วนชิ้นงานขนาดปานกลางและขนาดเล็กที่เหลือจะสะสมคะแนนรวมกันได้ทั้งหมด xx Story Points\enskip รายละเอียดคะแนนความยากของแต่ละงานจะถูกบันทึกไว้ในภาคผนวก

\[Insert Pie Chart Here\]

\section{ช่วงระยะเวลาการทำงาน}ข้าพเจ้าได้เริ่มต้นทำงานที่ฟีเจอร์ xPlatform เป็นระยะเวลาประมาณ 3 เดือน โดยในช่วงเวลาดังกล่าว ข้าพเจ้าได้ทำงานร่วมกับทีม Fast Easy เป็นระยะเวลา 1 สัปดาห์หลังจากเสร็จสิ้นโปรเจกต์แรก แลในระยะเวลาที่เหลือข้าพเจ้าได้มีส่วนร่วมในการทำงานในโปรเจกต์ขนาดกลางกับทีม xPlatform 