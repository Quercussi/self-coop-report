\chapter{\ifenglish Work Result\else ผลของการทำงานและสภาพแวดล้อมของการทำงานในองค์กร\fi}

\section{สรุปผลการทำงานตามรางขอบเขตงาน}
TODO: check accumulative story points
ตลอดระยะเวลาการทำงานที่บริษัท เอสซีบี เทคเอกซ์ ข้าพเจ้าได้สะสม Story Points รวมทั้งหมด xx คะแนน ซึ่งมากกว่าข้อกำหนดขั้นต่ำที่ระบุในรางขอบเขตงานที่ 60 Story Points อย่างไรก็ตาม ข้าพเจ้าไม่ได้มีโอกาสทำโปรเจคให้กับทีม SCB Easy ตามที่คาดหวังไว้ เนื่องจากทั้งสองทีมเห็นตรงกันว่าการย้ายข้าพเจ้าไปทำงานในทีม SCB Easy อาจไม่เหมาะสมมากนัก เนื่องจากข้าพเจ้าและเพื่อนร่วมฝึกงานยังไม่มีประสบการณ์กับโค้ดเบสของทีมนั้น ในขณะที่มีประสบการณ์ในการพัฒนาบนโปรเจค xPlatform มากพอที่จะเริ่มงานอื่น ๆ ได้เลย

อย่างไรก็ตาม ข้าพเจ้ายังคงได้ทำงานร่วมกับทีม SCB Easy ในระดับเล็กน้อย ส่วนงานที่เหลือจะเป็นงานย่อยขนาดกลางในทีม xPlatform ตามที่ได้ระบุไว้ในบทก่อนหน้า

\section{สัดส่วนการทำงาน}
Story Points ส่วนใหญ่จะอยู่ที่ฟีเจอร์ xPlatform Change Runbook ซึ่งมีคะแนนรวมถึง 42 Story Points เนื่องจากโครงการนี้เป็นโครงการใหญ่ตามที่กำหนดในรางขอบเขตงาน ในโครงการจะมีการแบ่งงานออกเป็นงานย่อย โดยเฉลี่ยแต่ละงานจะมีคะแนนประมาณ 3 Story Points ส่วนชิ้นงานขนาดปานกลางและขนาดเล็กที่เหลือจะสะสมคะแนนรวมกันได้ทั้งหมด xx Story Points\enskip รายละเอียดคะแนนความยากของแต่ละงานจะถูกบันทึกไว้ในภาคผนวก

TODO: recheck about this
\begin{table}[H]
    \centering
    \begin{tabular}{c||c|c}
        & \attr{Story Points} & \attr{อัตราส่วน} \\
        \hline\hline
        \attr{Change Runbook} & 42 & 42 \% \\
        \attr{User Management} & 20 & 41\% \\
        \attr{Custom Library} & 20 & 41\% \\
        \attr{Documentation} & 20 & 41\% \\
        \attr{Database Configration} & 1 & 5\% \\
    \end{tabular}
    \caption{ตารางแสดงสัดส่วนของ Story Points}
    \label{tab:story-point-table}
\end{table}

\begin{figure} [H]
    \begin{center}
        \begin{tikzpicture}
            \pie[radius=2]{41/Change Runbook,
                20/User Management,
                20/Custom Library,
                18/Documentation,
                1/Database Configuration
                }
        \end{tikzpicture}
    \end{center}
    \caption[แผนภูมิรูปวงกลมแสดงสัดส่วนของ Story Points]{แผนภูมิรูปวงกลมแสดงสัดส่วนของ Story Points}
    \label{fig:story-point-pie-chart}
\end{figure}

\section{ช่วงระยะเวลาการทำงาน}ข้าพเจ้าได้เริ่มต้นทำงานที่ฟีเจอร์ xPlatform เป็นระยะเวลาประมาณ 3 เดือน โดยในช่วงเวลาดังกล่าว ข้าพเจ้าได้ทำงานร่วมกับทีม SCB Easy เป็นระยะเวลา 1 สัปดาห์หลังจากเสร็จสิ้นโปรเจกต์แรก และในระยะเวลาที่เหลือข้าพเจ้าได้มีส่วนร่วมในการทำงานในโปรเจกต์ขนาดกลางกับทีม xPlatform ซึ่งจะได้แผนปฏิบัติงานสหกิจศึกษาดังนี้

\newcommand{\gantttitlevertical}[1]{\gantttitle{\rotatebox{90}{#1}}{1}}
\newcommand{\gantttitles}[2]{%
  \foreach \x in {#1} {%
    \gantttitle{\x}{#2}%
  }%
}
\newcommand{\blackcell}{
    \ganttbar[bar/.append style={fill opacity=1, pattern=crosshatch, pattern color=black},progress=100,inline=false]{}{16}{16}}

% link: https://tex.stackexchange.com/a/579761
\begin{table}[H]
    \centering
    \begin{ganttchart}[
        y unit title=0.5cm,
        y unit chart=0.6cm,
        x unit =0.6cm,
        % vgrid={draw=lightgray,line width=0.2pt},
        % hgrid={draw=lightgray,line width=0.2pt},
        vgrid,
        hgrid,
        title height=1,
        title/.style={fill=none, draw=black, line width=0.2pt},
        title label font=\footnotesize,
        bar/.style={fill=gray, fill opacity=0.5},
        bar height=1,
        bar top shift=0,
        progress label text={},
        group right shift=0,
        group height=.6,
        group peaks width={0.2},
        inline, 
        bar label node/.style={text width=2.75cm,
                               align=right,
                               anchor=east,
                               font=\small}
       ]{1}{16}
   
    \gantttitle{2024}{16}\\
                 
    \gantttitle{ก.ค.}{4} \gantttitle{ส.ค.}{4} \gantttitle{ก.ย.}{4} \gantttitle{ต.ค.}{4}\\

    \gantttitles{1,2,3,4,1,2,3,4,1,2,3,4,1,2,3,4}{1}\\
    
    \ganttbar[progress=100,inline=false]
        {Change Runbook}{1}{9}
    \ganttbar[progress=100,inline=false]
        {}{11}{11}
        \\
    \ganttbar[progress=100,inline=false]
        {DB Configuration}{10}{10}
        \\
    \ganttbar[progress=100,inline=false]
        {Documentation}{12}{12}
    \ganttbar[progress=100,inline=false]
        {}{13}{13}
        \\ 
    \ganttbar[progress=100,inline=false]
        {Custom Library}{12}{14} 
        \\
    \ganttbar[progress=100,inline=false]
        {User Management}{12}{12}
    \ganttbar[progress=100,inline=false]
        {}{14}{16} 
        \\
     \ganttbar[progress=100,inline=false]
        {Others}{10}{11} 
    \ganttbar[progress=100,inline=false]
        {}{13}{14} 

    \end{ganttchart}
    \caption{ตารางแผนปฏิบัติงานสหกิจศึกษา}
    \label{tab:work-timeline}
\end{table}

\section{\ifenglish Before Internship\else ก่อนที่จะปฏิบัติงานสหกิจศึกษา\fi}
ก่อนที่จะได้ปฏิบัติงานสหกิจศึกษานั้น ข้าพเจ้าได้ยื่นสมัครไป 3 บริษัท ได้แก่ บริษัท อะแวร์ คอร์ปอเรชั่น จำกัด ในตำแหน่ง Java Developer บริษัท เอสซีบี เทคเอกซ์ ในตำแหน่ง Software Engineer และบริษัท ที.ซี.ซี. เทคโนโลยี จำกัด ในตำแหน่ง Software Engineer 

โดยที่แต่ละบริษัทจะมีขั้นตอนการสมัครงานดังนี้
\begin{enumerate}
    \item บริษัท อะแวร์ คอร์ปอเรชั่น จำกัด: 
    \begin{enumerate}
        \item เริ่มต้นด้วยการส่งจดหมายสมัครงานทางอีเมลถึงฝ่ายบุคคล โดยในอีเมลต้องแนบ ใบรับรองผลการศึกษา (Transcript) ประวัติย่อ (Resume) และวิดีโอแนะนำตัวเอง
        \item ทำการสอบสัมภาษณ์ออนไลน์เกี่ยวกับการแก้ปัญหาเบื้องต้น และความรู้พื้นฐานเกี่ยวกับภาษา Java ใช้เวลาทั้งหมด 1 ชั่วโมง
        \item เข้าร่วมสัมภาษณ์สหกิจศึกษา โดยจะสอบถามเกี่ยวกับโปรเจคที่เคยทำ และให้อธิบายคำตอบจากการสอบครั้งก่อน ใช้เวลาในการสัมภาษณ์ประมาณ 1 ชั่วโมง
    \end{enumerate}
    \item บริษัท เอสซีบี เทคเอกซ์
    \begin{enumerate}
        \item ส่งแบบฟอร์ม Google Form ซึ่งประกอบไปด้วยข้อมูลส่วนตัวอย่างละเอียด เช่น ประวัติการศึกษา ผลการเรียน Resume และตำแหน่งที่สนใจในการทำงาน
        \item ทำการสอบออนไลน์เกี่ยวกับการแก้ไขปัญหาเบื้องต้นผ่านแพลตฟอร์ม HackerRank
        \item เข้าร่วมสัมภาษณ์สหกิจศึกษา ซึ่งในระหว่างการสัมภาษณ์นี้จะมีการสอบถามเกี่ยวกับโปรเจคที่เคยทำ รวมถึงการให้ผู้สมัครอธิบายคำตอบจากการสอบครั้งก่อน
    \end{enumerate}
\end{enumerate}
ทั้งบริษัท อะแวร์ คอร์ปอเรชั่น จำกัด และบริษัท เอสซีบี เทคเอกซ์ ได้ติดต่อข้าพเจ้าเพื่อเข้าฝึกงานอย่างเป็นทางการ ส่วนบริษัท ที.ซี.ซี. เทคโนโลยี จำกัด ได้ติดต่อครั้งแรกหลังจากที่ข้าพเจ้ายืนยันการฝึกปฏิบัติงานสหกิจศึกษาที่บริษัท เอสซีบี เทคเอกซ์ ทำให้ข้าพเจ้าไม่มีโอกาสในการสมัครงานกับบริษัทดังกล่าว
\begin{table}[H]
    \centering
    \begin{tabular}{c||c}
        \textbf{วันที่} & \textbf{เหตุการณ์} \\ \hline
        13 ม.ค. 2567 & ส่งจดหมายสมัครบริษัท อะแวร์ \\ \hline
        16 ม.ค. 2567 & บริษัท อะแวร์ ส่งจดหมายเชิญชวนทดสอบความรู้ออนไลน์ \\\hline
        18 ม.ค. 2567 & กรอกแบบฟอร์มสมัครบริษัทเอสซีบี เทคเอกซ์ \\ \hline
        30 ม.ค. 2567 & ทดสอบความรู้ออนไลน์กับบริษัท อะแวร์ \\\hline
        30 ม.ค. 2567 & บริษัท เอสซีบี เทคเอกซ์ ส่งจดหมายเชิญชวนทดสอบความรู้ออนไลน์ \\\hline
        31 ม.ค. 2567 & ทดสอบความรู้ออนไลน์กับบริษัท เอสซีบี เทคเอกซ์ \\\hline
        2 ก.พ. 2567 & บริษัท เอสซีบี เทคเอกซ์ ส่งจดหมายเชิญสัมภาษณ์ \\ \hline
        2 ก.พ. 2567 & บริษัท อะแวร์ ส่งจดหมายเชิญสัมภาษณ์ \\ \hline
        8 ก.พ. 2567 & สัมภาษณ์กับบริษัท เอสซีบี เทคเอกซ์ \\ \hline
        9 ก.พ. 2567 & สัมภาษณ์กับบริษัท อะแวร์ \\ \hline
        12 ก.พ. 2567& บริษัท อะแวร์ติดต่อแสดงความยินดีเข้าโปรแกรมฝึกงาน \\ \hline
        13 ก.พ. 2567& บริษัท เอสซีบี เทคเอกซ์ติดต่อแสดงความยินดีเข้าโปรแกรมฝึกงาน \\
    \end{tabular}
    \caption{ตารางแสดงไทม์ไลน์การหางาน}
    \label{tab:job-finding-timeline}
\end{table}

\section{\ifenglish Salary and Benefits \else เงินเดือนและสวัสดิการ \fi}
พนักงานทุกคนจะได้รับเงินเดือนเริ่มต้นที่ 30,000 บาท โดยเงินเดือนจะปรับขึ้นทุกปีตามผลการปฏิบัติงาน สำหรับนักศึกษาฝึกงานหรือสหกิจศึกษา จะได้รับเบี้ยเลี้ยงวันละ 500 บาท คิดเป็นประมาณ 10,000 บาทต่อเดือน

นอกจากเงินเดือนและเบี้ยเลี้ยงแล้ว พนักงานยังได้รับสิทธิประโยชน์เพิ่มเติมดังนี้:
\begin{enumerate}
    \item พนักงานแต่ละคนจะได้รับ MacBook Pro สำหรับใช้งานตลอดระยะเวลาการทำงาน โดยต้องส่งคืนเมื่อสิ้นสุดการจ้างงาน
    \item อาหารเช้าฟรีทุกวันทำงาน
    \item ในแต่ละปีพนักงานสามารถลาพักร้อน 15 วัน ลาเดือนเกิด 1 วัน ลากิจ 5 วัน และลาป่วย 30 วัน โดยสามารถสะสมวันลาพักร้อนข้ามปีได้สูงสุดถึง 5 วัน นอกจากนี้พนักงานสามารถลาบวช 105 วัน ซึ่งไม่เปลี่ยนแปลงในแต่ละปี
    \item นักศึกษาฝึกงานจะได้รับประกันชีวิตจำนวน 5000 บาท ประกันอุบัติเหตุ 150000 บาท และความคุ้มครองค่ารักษาพยาบาล (AME) อีก 5000 บาท โดยทั้งหมดเป็นแบบประกันกลุ่ม
    \item พนักงานทั่วไปจะได้รับประกันชีวิตจำนวน 1000000 บาท ประกันอุบัติเหตุจำนวน 1000000 บาท และประกันสุขภาพตามแผนที่พนักงานเลือก
    \item บริการรถรับส่งพนักงานระหว่างสำนักงานและสถาที่ต่าง ๆ สองรอบต่อวัน:
    \begin{enumerate}
        \item รอบเช้า: 6:30 – 9:00 น.
        \item รอบเย็น: 17:20 – 20:00 น.
    \end{enumerate}
    จุดขึ้น-ลงรถมี 4 สถานที่:
    \begin{enumerate}
        \item อาคาร SCB Plaza West
        \item BTS หมอชิต (ทางออก 2)
        \item MRT สวนจตุจักร (ทางออก 4)
        \item Big C วงศ์สว่าง
    \end{enumerate}
\end{enumerate}


\section{\ifenglish Corporate Culture \else วัฒนธรรมองค์กร \fi} 
บริษัทใช้รูปแบบการทำงานแบบไฮบริด 8 ชั่วโมงต่อวัน 9:00 - 18.00 น. มีพักเบรกรับประทานอาหารเที่ยง 1 ชั่วโมง โดยพนักงานจะเข้ามาทำงานที่สำนักงาน 2 วันต่อสัปดาห์ และทำงานทางออนไลน์อีก 3 วัน ส่วนวันที่ต้องเข้ามาทำงานที่สำนักงานจะมีการกำหนดให้แต่ละทีมผลัดกันมา

\subsection{กิจกรรมและการอบรม}
บริษัทมีการจัดกิจกรรมที่เปิดให้พนักงานทุกคนสามารถเข้าร่วมได้ เพื่อส่งเสริมการการพัฒนาทักษะต่าง ๆ ที่เกี่ยวข้องกับการทำงาน กิจกรรมเหล่านี้ยังช่วยสร้างความสัมพันธ์ระหว่างพนักงานและผู้บริหาร โดยมีตัวอย่างกิจกรรมดังต่อไปนี้
\begin{enumerate}
    \item Knowledge Sharing: กิจกรรมที่ให้โอกาสที่ผู้มีความรู้และประสบการณ์มาร่วมพูดคุยและแบ่งปันความรู้ใหม่ ๆ เช่น เทคนิคการออกแบบ UX/UI ให้ประสบคสามสำเร็จ วิธีป้องกันตัวเองจาก Office Syndrome และการดูแลสุขภาพร่างกาย รวมถึงเคล็ดลับในการเพิ่มความสุขและสร้างชีวิตที่มีความสุขยิ่งขึ้นในที่ทำงาน
    \item Town Hall: กิจกรรมที่ CEO จะมาพบปะพูดคุยและตอบคำถามจากพนักงานในบริษัท โดยกิจกรรมนี้จะจัดขึ้นทุก ๆ 2 เดือน
    \item กิจกรรมวิ่งมาราธอนเฉลิมฉลองครบรอบ 3 ปี
\end{enumerate}
สำหรับนักศึกษาฝึกงานจะมีการอบรมที่จัดขึ้นโดยแผนก Human Resources เพื่อเสริมสร้างความรู้ที่สำคัญต่อการทำงาน นอกเหนือจากการพัฒนาซอฟต์แวร์อย่างเดียว โดยกิจกรรมเหล่านี้จะช่วยให้นักศึกษาได้เรียนรู้ทักษะใหม่ ๆ และเข้าใจเกี่ยวกับการทำงานในสภาพแวดล้อมจริง
\begin{enumerate}
    \item \textbf{Welcome New Intern: }กิจกรรมแนะนำบริษัท เอสซีบี เทคเอกซ์ และสถานที่ทำงานที่ SCBX Park Plaza เพื่อให้นักศึกษาฝึกงานได้รู้จักกับองค์กรและสภาพแวดล้อมการทำงาน

    \item \textbf{Sharing Session: }กิจกรรมที่พนักงานจากแต่ละตำแหน่งมาร่วมแบ่งปันประสบการณ์และรายละเอียดเกี่ยวกับงานที่ทำในตำแหน่งนั้น ๆ ให้นักศึกษาฝึกงานได้เรียนรู้
    
    \item \textbf{Tips on Building a Strong Profile: }กิจกรรมอบรมการเขียนเรซูเม่ การจัดการโปรไฟล์โซเชียลมีเดียให้น่าสนใจ และการสร้างการเชื่อมต่อกับบุคคลอื่น
    
    \item \textbf{Agile \& Cloud \& AI Skills: }กิจกรรมอบรมเกี่ยวกับการใช้งาน Azure Cloud Services และเทคนิค Agile ที่เกี่ยวข้องกับการพัฒนาซอฟต์แวร์
    
    \item \textbf{Intern Checking: }กิจกรรมผ่อนคลายสำหรับนักศึกษาฝึกงานให้ได้มีโอกาสสร้างสัมพันธ์และพูดคุยกัน
    
    \item \textbf{Design Thinking: }กิจกรรมอบรมการคิดเชิงออกแบบ (Design Thinking) สำหรับการพัฒนาซอฟต์แวร์
    
    \item \textbf{Communication Skills: }กิจกรรมอบรมเกี่ยวกับทักษะการสื่อสารและการพูดคุยกับผู้อื่น
    
\end{enumerate}

\subsection{ความคิดเห็นส่วนตัว}
TODO 

\section{ข้อเสนอแนะในการสมัครงาน}
ในส่วนของทีม xPlatform Developer นั้น สมาชิกในทีมทุกคนมีทักษะแบบ Full-stack และมีความรู้ในการรัน Jenkins Jobs ด้วยเช่นกัน โดยทักษะที่สำคัญประกอบด้วย:

\begin{enumerate}
    \item \textbf{Front-end: } ต้องมีความรู้ในการใช้ React NextJS และ Apollo Client
    \item \textbf{Back-end: } ควรมีประสบการณ์ในการพัฒนาโปรเจคแบบ Microservice ใช้ Typescript และ ExpressJS ได้อย่างคล่องแคล่ว รวมถึงมีความรู้ใน Apollo GraphQL
\end{enumerate}

อย่างไรก็ตาม มันไม่มีไม่จำเป็นต้องเชี่ยวชาญทุกด้าน ตัวอย่างเช่น ข้าพเจ้าไม่มีประสบการณ์ในการเขียน backend ด้วย Typescript ไม่เคยพัฒนา Microservice หรือไม่เคยใช้งาน GraphQL มาก่อน แต่สิ่งที่ผู้สมัครควรที่จะให้ความสำคัญคือการแสดงให้เห็นถึงความรู้ในเครื่องมือและเทคโนโลยีต่าง ๆ ในอุตสาหกรรมซอฟต์แวร์ รวมถึงความพร้อมที่จะเรียนรู้สิ่งใหม่ ๆ

\section{ข้อเสนอจากบริษัท}
ทางบริษัทไม่ได้ให้ข้อเสนอการทำงานทันทีหลังจากที่นักศึกษาจบการศึกษาในมหาวิทยาลัย อย่างไรก็ตามจากการพูดคุยกับผู้ที่เกี่ยวข้อง มีการกล่าวว่าผู้ที่ผ่านการปฏิบัติงานสหกิจศึกษาหรือฝึกงานกับบริษัทมีโอกาสค่อนข้างสูงที่จะได้รับการพิจารณาในการรับเข้าทำงานหลังจากสำเร็จการศึกษา แม้ว่าจะยังไม่มีการรับรองอย่างเป็นทางการในเรื่องนี้

