\chapter{\ifenglish Work environment\else สภาพแวดล้อมของการทำงานในองค์กร\fi}

\section{\ifenglish Before Internship\else ก่อนที่จะปฏิบัติงานสหกิจศึกษา\fi}
ก่อนที่จะได้ปฏิบัติงานสหกิจศึกษานั้น ข้าพเจ้าได้ยื่นสมัครไป 3 บริษัท ได้แก่ บริษัท อะแวร์ คอร์ปอเรชั่น จำกัด ในตำแหน่ง Java Developer บริษัท เอสซีบี เทคเอกซ์ ในตำแหน่ง Software Engineer และบริษัท ที.ซี.ซี. เทคโนโลยี จำกัด ในตำแหน่ง Software Engineer 

โดยที่แต่ละบริษัทจะมีขั้นตอนการสมัครงานดังนี้
\begin{enumerate}
    \item บริษัท อะแวร์ คอร์ปอเรชั่น จำกัด: 
    \begin{enumerate}
        \item เริ่มต้นด้วยการส่งจดหมายสมัครงานทางอีเมลถึงฝ่ายบุคคล โดยในอีเมลต้องแนบ ใบรับรองผลการศึกษา (Transcript) ประวัติย่อ (Resume) และวิดีโอแนะนำตัวเอง
        \item ทำการสอบสัมภาษณ์ออนไลน์เกี่ยวกับการแก้ปัญหาเบื้องต้น และความรู้พื้นฐานเกี่ยวกับภาษา Java ใช้เวลาทั้งหมด 1 ชั่วโมง
        \item เข้าร่วมสัมภาษณ์สหกิจศึกษา โดยจะสอบถามเกี่ยวกับโปรเจคที่เคยทำ และให้อธิบายคำตอบจากการสอบครั้งก่อน ใช้เวลาในการสัมภาษณ์ประมาณ 1 ชั่วโมง
    \end{enumerate}
    \item บริษัท เอสซีบี เทคเอกซ์
    \begin{enumerate}
        \item ส่งแบบฟอร์ม Google Form ซึ่งประกอบไปด้วยข้อมูลส่วนตัวอย่างละเอียด เช่น ประวัติการศึกษา ผลการเรียน Resume และตำแหน่งที่สนใจในการทำงาน
        \item ทำการสอบออนไลน์เกี่ยวกับการแก้ไขปัญหาเบื้องต้นผ่านแพลตฟอร์ม HackerRank
        \item เข้าร่วมสัมภาษณ์สหกิจศึกษา ซึ่งในระหว่างการสัมภาษณ์นี้จะมีการสอบถามเกี่ยวกับโปรเจคที่เคยทำ รวมถึงการให้ผู้สมัครอธิบายคำตอบจากการสอบครั้งก่อน
    \end{enumerate}
\end{enumerate}
ทั้งบริษัท อะแวร์ คอร์ปอเรชั่น จำกัด และบริษัท เอสซีบี เทคเอกซ์ ได้ติดต่อข้าพเจ้าเพื่อเข้าฝึกงานอย่างเป็นทางการ ส่วนบริษัท ที.ซี.ซี. เทคโนโลยี จำกัด ได้ติดต่อมาในภายหลัง หลังจากที่ข้าพเจ้ายืนยันการฝึกปฏิบัติงานสหกิจศึกษาที่บริษัท เอสซีบี เทคเอกซ์ ทำให้ข้าพเจ้าไม่มีโอกาสในการสมัครงานกับบริษัทดังกล่าว
\begin{table}[H]
    \centering
    \begin{tabular}{c||c}
        \textbf{วันที่} & \textbf{เหตุการณ์} \\ \hline
        13 ม.ค. 2567 & ส่งจดหลายสมัครบริษัท อะแวร์ \\ \hline
        16 ม.ค. 2567 & บริษัท อะแวร์ ส่งจดหมายเชิญชวนทดสอบความรู้ออนไลน์ \\\hline
        30 ม.ค. 2567 & ทดสอบความรู้ออนไลน์ กับบริษัท อะแวร์ \\\hline
        2 ก.พ. 2567 & บริษัท เอสซีบี เทคเอกซ์ ส่งจดหมายเชิญชวนทดสอบความรู้ออนไลน์ \\ \hline
        2 ก.พ. 2567 & บริษัท อะแวร์ ส่งจดหมายเชิญสัมภาษณ์ \\ \hline
        9 ก.พ. 2567 & ทำสัมภาษณ์กับบริษัท อะแวร์ \\ \hline
        TODO& EXTEND THIs \\ \hline
        5 มี.ค. 2567& บริษัท เอสซีบี เทคเอกซ์ส่งจดหมายแสดงความยินดีเข้าโปรแกรมฝึกงาน \\
    \end{tabular}
    \caption{ตารางแสดงไทม์ไลน์การหางาน}
    \label{tab:job-finding-timeline}
\end{table}

\section{\ifenglish Salary and Benefits \else เงินเดือนและสวัสดิการ \fi}
พนักงานทุกคนจะได้รับเงินเดือนเริ่มต้นที่ 30,000 บาท โดยเงินเดือนจะปรับขึ้นทุกปีตามผลการปฏิบัติงาน สำหรับนักศึกษาฝึกงานหรือสหกิจศึกษา จะได้รับเบี้ยเลี้ยงวันละ 500 บาท คิดเป็นประมาณ 10,000 บาทต่อเดือน

นอกจากเงินเดือนและเบี้ยเลี้ยงแล้ว พนักงานยังได้รับสิทธิประโยชน์เพิ่มเติมดังนี้:
\begin{enumerate}
    \item พนักงานแต่ละคนจะได้รับ MacBook Pro สำหรับใช้งานตลอดระยะเวลาการทำงาน โดยต้องส่งคืนเมื่อสิ้นสุดการจ้างงาน
    \item อาหารเช้าฟรีทุกวันทำงาน
    \item ในแต่ละปีพนักงานสามารถลาพักร้อน 15 วัน ลาเดือนเกิด 1 วัน ลากิจ 5 วัน และลาป่วย 30 วัน โดยสามารถสะสมวันลาพักร้อนข้ามปีได้สูงสุดถึง 5 วัน นอกจากนี้พนักงานสามารถลาบวช 105 วัน ซึ่งไม่เปลี่ยนแปลงในแต่ละปี
    \item นักศึกษาฝึกงานจะได้รับประกันชีวิตจำนวน 5000 บาท ประกันอุบัติเหตุ 150000 บาท และความคุ้มครองค่ารักษาพยาบาล (AME) อีก 5000 บาท โดยทั้งหมดเป็นแบบประกันกลุ่ม
    \item พนักงานทั่วไปจะได้รับประกันชีวิตจำนวน 1000000 บาท ประกันอุบัติเหตุจำนวน 1000000 บาท และประกันสุขภาพตามแผนที่พนักงานเลือก
    \item บริการรถรับส่งพนักงานระหว่างสำนักงานและสถาที่ต่าง ๆ สองรอบต่อวัน:
    \begin{enumerate}
        \item รอบเช้า: 6:30 – 9:00 น.
        \item รอบเย็น: 17:20 – 20:00 น.
    \end{enumerate}
    จุดขึ้น-ลงรถมี 4 สถานที่:
    \begin{enumerate}
        \item อาคาร SCB Plaza West
        \item BTS หมอชิต (ทางออก 2)
        \item MRT สวนจตุจักร (ทางออก 4)
        \item Big C วงศ์สว่าง
    \end{enumerate}
\end{enumerate}


\section{\ifenglish Corporate Culture \else วัฒนธรรมองค์กร \fi} 
บริษัทใช้รูปแบบการทำงานแบบไฮบริด 8 ชั่วโมงต่อวัน 9:00 - 18.00 น. มีพักเบรกรับประทานอาหารเที่ยง 1 ชั่วโมง โดยพนักงานจะเข้ามาทำงานที่สำนักงาน 2 วันต่อสัปดาห์ และทำงานทางออนไลน์อีก 3 วัน ส่วนวันที่ต้องเข้ามาทำงานที่สำนักงานจะมีการกำหนดให้แต่ละทีมผลัดกันมา

\subsection{กิจกรรมและการอบรม}
ในบริษัทจะมีการจัดกิจกรรมบ้างอย่างเช่น
\begin{enumerate}
    \item 
\end{enumerate}
และสำรับนักศึกษาฝึกงนก็จะมีกิจกรรมที่ต้องเข้าร่วมแบบ TODO: filthis
\begin{enumerate}
    \item 
\end{enumerate}
และก็มีกิจกรรมอีกมากมาย (TODO: report that i didn't listed all)

\subsection{ความคิดเห็นส่วนตัว}

\section{ข้อเสนอจากบริษัท}
ทางบริษัทไม่ได้ให้ข้อเสนอการทำงานทันทีหลังจากที่นักศึกษาจบการศึกษาในมหาวิทยาลัย อย่างไรก็ตามจากการพูดคุยกับผู้ที่เกี่ยวข้อง มีการกล่าวว่าผู้ที่ผ่านการปฏิบัติงานสหกิจศึกษาหรือฝึกงานกับบริษัทมีโอกาสค่อนข้างสูงที่จะได้รับการพิจารณาในการรับเข้าทำงานหลังจากสำเร็จการศึกษา แม้ว่าจะยังไม่มีการรับรองอย่างเป็นทางการในเรื่องนี้

