\chapter{\ifenglish Introduction\else บทนำ\fi}

\section{\ifenglish The Importance and Origin of the 
project\else ความสำคัญและที่มาของโครงงาน\fi}

% Okay there is actaully a tab down here vvv
การจัดเก็บภาษีถือเป็นหนึ่งในรายได้หลักของประเทศไม่ว่าจะเป็นภาษีทางตรง อย่างเช่น ภาษีทางตรง ภาษีรายได้บุคคลธรรมดา ซึ่งจะจัดเก็บได้
จากประชาชนผู้มีเงินได้ทั่วไป ภาษีเงินได้นิติบุคคลซึ่งเป็นภาษีที่จัดเก็บได้จากเงินได้ของบริษัทหรือห้างหุ้นส่วนนิติบุคคล และยังมีภาษีทางอ้อม เช่น 
ภาษีมูลค่าเพิ่ม ภาษีธุรกิจเฉพาะ ซึ่งเงินที่ได้จากการเก็บภาษีเหล่าล้วนนำไปให้รัฐบาลใช้ในการพัฒนาประเทศให้เจริญก้าวหน้า  

ภาษีป้ายก็เป็นส่วนหนึ่งของรายได้ท้องถิ่นที่สามารถจัดเก็บได้โดยองค์กรปกครองส่วนท้องถิ่น โดยที่ภาษีลักษณะนี้เมื่อจัดเก็บได้แล้ว ทางท้องถิ่น
ไม่จำเป็นต้องส่งคืนให้ทางรัฐ สามารถนำไปใช้จัดการบริหารพัฒนาภายในท้องถิ่นของตนเองได้ 
แต่ด้วยความสามารถในการจัดเก็บภาษีป้ายขององค์กรปกครองส่วนท้องถิ่นในแต่ละที่ ขึ้นอยู่กับปัจจัยหลาย ๆ อย่าง เช่น 
การที่ไม่สามารถรู้ได้ว่าป้ายที่สามารถจัดเก็บภาษีได้นั้นอยู่ที่ตำแหน่งใดในเขตปกครอง ซึ่งมีส่วนทำให้ประสิทธิภาพในการค้นหาป้ายภายในท้องถิ่นที่มีอยู่ทำได้อยู่จำกัด
และเป็นขั้นตอนที่ต้องใช้กำลังคนในการตรวจสอบเป็นอย่างมาก 
ดังนั้นจากปัญหาในจุดที่กล่าวมาทำให้เกิดโครงงานที่เป็นเครื่องมือที่ช่วยในการตรวจจับหาป้ายที่คาดว่าจะสามารถนำไปจัดเก็บภาษี 
และรายงานผลให้กับแต่ละองค์กรปกครองส่วนท้องถิ่นให้ไปจัดเก็บภาษีจากป้ายเหล่านี้

\section{\ifenglish The Objective of the Study\else วัตถุปรสงค์ของการศึกษา\fi}
\begin{enumerate}
    \item ได้ประยุกต์ใช้ทักษะการพัฒนาซอฟต์แวร์ที่ในศึกษาในมหาวิทยาลัยและแหล่งอื่น ๆ บนโครงงานที่อยู่ในอุตสาหกรรมซิฟต์แวร์จริง
    \item ได้ประสบการณ์ในการพัฒนาเว็บแอปพลิเคชันในอุตสาหกรรม
    \item ได้ซึกษาระบบการทำงานของการเป็นผู้พัฒนาซอฟต์แวร์ รวมไปถึงเทคโนโลยีต่าง ๆ ที่ช่วยในการพัฒนาซอฟต์แวร์
\end{enumerate}