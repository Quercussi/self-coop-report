\chapter{\ifenglish Introduction\else บทนำ\fi}

\section{\ifenglish The Importance and Origin of the 
project\else ความสำคัญและที่มาของโครงงาน\fi}
TODO

\section{\ifenglish The Objective of the Study\else วัตถุประสงค์ของการศึกษา\fi}
\begin{enumerate}
    \item ได้ประยุกต์ใช้ทักษะการพัฒนาซอฟต์แวร์ที่ในศึกษาในมหาวิทยาลัยและแหล่งอื่น ๆ บนโครงงานที่อยู่ในอุตสาหกรรมซิฟต์แวร์จริง
    \item ได้ประสบการณ์ในการพัฒนาเว็บแอปพลิเคชันในอุตสาหกรรม
    \item ได้ศึกษาระบบการทำงานของการเป็นผู้พัฒนาซอฟต์แวร์ รวมไปถึงเทคโนโลยีและอุปกรณ์ต่าง ๆ ที่ช่วยในการพัฒนาซอฟต์แวร์
\end{enumerate}

\section{\ifenglish Company Products and Services\else สินค้าและบริการของบริษัท \fi}
บริษัท เอสซีบี เทคเอกซ์ มีความเชี่ยวชาญในการพัฒนานวัตกรรมและเทคโนโลยีที่สามารถตอบสนองความต้องการของลูกค้าในด้านการบริการอย่างครบวงจร โดยนำเสนอบริการที่หลากหลายซึ่งครอบคลุมตั้งแต่การให้คำปรึกษาไปจนถึงการพัฒนาโซลูชันทางเทคโนโลยี รวมถึงการวิเคราะห์ความต้องการของระบบ การออกแบบ การพัฒนาซอฟต์แวร์จนถึงการใช้งานจริง นอกจากนี้ บริษัทยังให้บริการด้านโครงสร้างพื้นฐานเทคโนโลยีและการประมวลผลบนระบบคลาวด์เพื่อสนับสนุนการดำเนินงานในยุคดิจิทัล

เพื่อสนับสนุนการตัดสินใจเชิงธุรกิจ บริษัทได้พัฒนาบริการจัดการข้อมูลที่สามารถสร้างข้อมูลเชิงลึกให้กับลูกค้า พร้อมทั้งมีบริการด้านความปลอดภัยทางไซเบอร์เพื่อปกป้องข้อมูลและระบบการดำเนินงานในสภาพแวดล้อมดิจิทัล


\begin{enumerate}
  \item eKYC (Electronic Know Your Customer) เป็นซอฟต์แวร์ที่พัฒนาขึ้นตามพระราชบัญญัติป้องกันและปราบปรามการฟอกเงิน ซึ่งกำหนดให้ธุรกิจที่เกี่ยวข้องกับการเงินและการลงทุนต้องดำเนินการระบบ KYC (Know Your Customer) ก่อนทำธุรกรรม ในอดีต การยืนยันตัวตนผู้ใช้บริการจะต้องใช้วิธีการกรอกเอกสาร ซึ่งอาจทำให้กระบวนการช้าและซับซ้อน เพื่อเพิ่มความสะดวกและความรวดเร็วในการยืนยันตัวตน บริษัทจึงได้สร้างซอฟต์แวร์นี้ขึ้น โดยนำเทคโนโลยีการสแกนใบหน้าและบัตรประชาชนมาใช้ ซึ่งช่วยให้กระบวนการยืนยันตัวตนสามารถดำเนินการได้อย่างรวดเร็ว
  \item xPlatform ได้นำแนวทางปฏิบัติที่มีประสิทธิภาพใน DevOps มาประยุกต์ใช้ในการออกแบบแพลตฟอร์มอัตโนมัติ  ในรูปแบบ Web Application ซึ่งช่วยให้ลดภาระการทำงานของทั้งทีมพัฒนาและทีมปฏิบัติการ โดยที่แพลตฟอร์มนี้มีฟีเจอร์ที่รองรับทุกขั้นตอนของวงจรซอฟต์แวร์ ตั้งแต่การพัฒนา การทดสอบ การปล่อยซอฟต์แวร์ ไปจนถึงการบำรุงรักษา การเฝ้าระวัง และการเพิ่มประสิทธิภาพ นอกจากนี้ยังช่วยสนับสนุนการทำงานร่วมกันแบบ Agile บนแพลตฟอร์มเดียว ซึ่งฟีเจอร์เหล่านี้ยังช่วยให้ PO/PM สามารถบริหารทีมและควบคุมงบประมาณของโครงการเพื่อลดค่าใช้จ่ายที่ไม่จำเป็น
  \item TechX Data Platform เป็นแพลตฟอร์มที่ออกแบบมาเพื่อให้การจัดการข้อมูลเป็นเรื่องง่ายและครบวงจร โดยครอบคลุมทุกขั้นตอน ตั้งแต่การนำเข้าข้อมูล การจัดเก็บ การจัดการ การวิเคราะห์ จนถึงการรักษาความปลอดภัยของข้อมูล นอกจากนี้ยังมีบริการวิเคราะห์ข้อมูลขั้นสูงด้วยเทคโนโลยี Machine Learning ที่สามารถปรับแต่งให้เหมาะสมกับธุรกิจได้ทุกรูปแบบ ไม่ว่าจะเป็น Startup ธุรกิจ SME หรือองค์กรขนาดใหญ่ ซึ่งมีความต้องการด้านข้อมูลที่แตกต่างกันไป
  \item บริษัทมีบริการให้คำปรึกษาและพัฒนาโซลูชันครบวงจร เพื่อตอบสนองความต้องการของลูกค้าและผู้มีส่วนเกี่ยวข้อง โดยใช้เฟรมเวิร์คกระบวนการคิดเชิงออกแบบ (Design Thinking) นอกจากนี้ บริษัทยังมีทีมวิศวกรซอฟต์แวร์และนักออกแบบประสบการณ์ผู้ใช้ (UX Designer) ที่มีความเชี่ยวชาญในการพัฒนาโซลูชันให้กลายเป็นผลิตภัณฑ์จริง
  \item บริการโซลูชันด้านคลาวด์ที่เน้นความยืดหยุ่นและประสิทธิภาพในการจัดการโครงสร้างพื้นฐานทางไอที รวมถึงการย้ายข้อมูลและการรักษาความปลอดภัยบนคลาวด์ บริการเหล่านี้ครอบคลุมตั้งแต่การออกแบบสถาปัตยกรรมระบบ การจัดการทรัพยากรไอที จนถึงการเฝ้าระวังและปรับปรุงประสิทธิภาพของระบบ ทีมงานยังมีความเชี่ยวชาญในการบริหารจัดการระบบคลาวด์หลากหลายแพลตฟอร์ม (multi-cloud) และใช้กระบวนการที่เน้นความปลอดภัยในทุกขั้นตอน
\end{enumerate}

\renewcommand{\arraystretch}{1.2}
\newcommand{\attr}[1]{\hspace{2pt}#1\hspace{2pt}}
\section{งบแสดงฐานะการเงิน}
\begin{table}[H]
  \centering
  \begin{tabular}{c||c|c|c}
      \multirow{2}{*}{ปี} & \multicolumn{3}{c}{จำนวน (ล้านบาท)} \\
      \cline{2-4}
       & \attr{สินทรัพย์} & \attr{หนี้สิน} & \attr{ส่วนผู้ถือหุ้น} \\
      \hline\hline
      2021 & 1742 & 1241 & 501 \\
      2022 & 2351 & 821  & 1529\\
      2023 & 1954 & 587  & 1367\\
  \end{tabular}
  \caption{โครงสร้างงบฐานะการเงินปี 2021 ถึงปี 2023}
  \label{tab:company-asset-table-1}
\end{table}
\begin{table}[H]
  \centering
  \begin{tabular}{c||c|c|c}
       ปี & \attr{สินทรัพย์} & \attr{หนี้สิน} & \attr{ส่วนผู้ถือหุ้น} \\
      \hline\hline
      2021 & 0 & 0  & 0 \\
      2022 & 0.35 & -0.34  & 2.05\\
      2023 & -0.11 & -0.29  & -0.17\\
  \end{tabular}
  \caption{อัตราการเปลี่ยนแปลงโครงสร้างงบฐานะการเงินปี 2021 ถึงปี 2023}
  \label{tab:company-asset-table-2}
\end{table}
บริษัท เอสซีบี เทคเอกซ์ ซึ่งเป็นบริษัทในเครือของธนาคารไทยพาณิชย์ มีสินทรัพย์ที่สูงมากสำหรับบริษัทใหม่ โดยในปีแรก (2021) บริษัทมีสินทรัพย์รวม 1742 ล้านบาท จากนั้นสินทรัพย์เพิ่มขึ้นเป็น 2351 ล้านบาทในปี 2022 ในปีนี้ ส่วนของผู้ถือหุ้นเพิ่มขึ้นกว่า 1000 ล้านบาท จาก 501 ล้านบาทในปี 2021 เป็น 1529 ล้านบาท แม้ในปี 2023 สินทรัพย์จะลดลงเล็กน้อยมาอยู่ที่ 1954 ล้านบาท แต่ส่วนของผู้ถือหุ้นยังคงสูงอยู่ที่ 1367 ล้านบาท

ในด้านหนี้สิน บริษัทเริ่มต้นด้วยหนี้สิน 1241 ล้านบาทในปี 2021 ทำให้อัตราส่วนหนี้สินต่อทุนอยู่ที่ 2.43 เท่า จากนั้นบริษัทปรับลดระดับหนี้สินอย่างต่อเนื่อง ในปี 2022 หนี้สินลดลงเหลือ 821 ล้านบาท และในปี 2023 ลดลงเหลือ 587 ล้านบาท ส่งผลให้อัตราส่วนหนี้สินต่อทุนในปี 2023 อยู่ที่ 0.49 เท่า

\section{งบกำไรขาดทุน}
\begin{table}[H]
  \centering
  \begin{tabular}{c||c|c|c|c}
      \multirow{2}{*}{ปี} & \multicolumn{4}{c}{จำนวน (ล้านบาท)} \\
      \cline{2-5}
       & \attr{กำไรสุทธิ} & \attr{กำไรก่อนภาษี} & \attr{รายจ่ายรวม} & \attr{รายได้รวม} \\
      \hline\hline
      2021 & 350 & 438 & 1158 & 1596\\
      2022 & 672 & 840  & 2329 & 3169\\
      2023 & 246 & 308  & 1978 & 2289\\ 
  \end{tabular}
  \caption{โครงสร้างงบกำไรขาดทุนปี 2021 ถึงปี 2023}
  \label{tab:company-asset-table-1}
\end{table}
\begin{table}[H]
  \centering
  \begin{tabular}{c||c|c|c|c}
       ปี & \attr{กำไรสุทธิ} & \attr{กำไรก่อนภาษี} & \attr{รายจ่ายรวม} & \attr{รายได้รวม} \\
      \hline\hline
      2021 & 0 & 0 & 0 & 0\\
      2022 & 0.92 & 0.92  & 1.01 & 0.99\\
      2023 & -0.63 & -0.63  & -0.15 & -0.28\\
  \end{tabular}
  \caption{อัตราการเปลี่ยนแปลงโครงสร้างงบกำไรขาดทุนปี 2021 ถึงปี 2023}
  \label{tab:company-asset-table-2}
\end{table}
ในปีแรกของบริษัท มีผลกำไรสุทธิอยู่ที่ 350 ล้านบาท ขณะที่รายได้รวมอยู่ที่ 1596 ล้านบาท ซึ่งแสดงให้เห็นว่าบริษัทมีอัตรากำไรสุทธิที่ 0.219 เท่า เมื่อเปรียบเทียบกับรายได้รวมในปีถัดมาในปี 2022 อัตราการเพิ่มของรายได้รวมและรายจ่ายรวมของบริษัทได้เพิ่มขึ้นเท่าตัว โดยรายจ่ายรวมอยู่ที่ 2329 ล้านบาท และรายได้รวมอยู่ที่ 3169 ล้านบาท ซึ่งอัตรากำไรสุทธิในปีนั้นอยู่ที่ 0.212 เท่า แสดงให้เห็นว่าบริษัทยังคงรักษาอัตรากำไรสุทธิในระดับที่ใกล้เคียงกับปีแรก

ในปีถัดมา มีการลดอัตรารายจ่ายรวมและรายได้รวม โดยอัตราการลดของรายได้รวมนั้นเกือบสองเท่าของอัตราการลดรายจ่ายรวม ทำให้รายจ่ายรวมลดลงเหลือ 1978 ล้านบาท และรายได้รวมอยู่ที่ 2289 ล้านบาท ผลลัพธ์นี้ทำให้อัตรากำไรสุทธิของบริษัทตกลงเหลือเพียง 0.107 เท่า ซึ่งแสดงให้เห็นว่าบริษัทมีการเติบโตในปีแรกและปีที่สอง แต่ในปีที่สามกลับมีแนวโน้มการลดลงของอัตรากำไรสุทธิ อาจบ่งบอกถึงความท้าทายที่บริษัทเผชิญหรือเกิดจากการตัดสินใจลดขนาดหนี้สิน