% TODO: UNCOMMENT THE FOLLOWING
% \chapter{\ifproject%
% \ifenglish Experimentation and Results\else การทดลองและผลลัพธ์\fi
% \else%
% \ifenglish System Evaluation\else การประเมินระบบ\fi
% \fi}

% TODO: DELETE THE FOLLOWING
\chapter{\ifenglish System Evaluation\else การประเมินระบบ\fi}

\section{การประเมินประสิทธิภาพซอฟต์แวร์}
ทดสอบประสิทธิภาพซอฟต์แวร์โดยจะมีการแบ่งส่วนในการทดสอบออกเป็นส่วน ๆ เพื่อให้รู้ว่าในแต่ละส่วนของซอฟต์แวร์ของเรานั้น 
ทำงานได้อย่างมีประสิทธิภาพหรือไม่ จึงสามารถแบ่งออกการประเมินได้เป็นดังนี้ 
\begin{enumerate}
    \item Classification model - เป็นการทดสอบเพื่อประเมินและตรวจสอบความเร็วในการประมวลผลเพื่อทำการ classify 
    ว่า object ใดเป็นป้ายที่สามารถจัดเก็บภาษีได้ รวมถึงในเรื่องของความแม่นยำในการ classify  
    \item Response time - เป็นการทดสอบเพื่อประเมินในเรื่องของความเร็วในการรับส่งข้อมูลระหว่าง client กับ application server  
\end{enumerate}

\section{การประเมินความพึงพอใจในการใช้งานระบบ}
ทดสอบความพึงพอใจในการใช้งานจะมีการแบ่งออกเป็นสองส่วน คือส่วนของแอปพลิเคชันในโทรศัพท์มือถือ กับส่วนของเว็บแอปพลิเคชัน 
โดยจะมีเกณฑ์การให้คะแนนอยู่ที่ 1 ถึง 5 โดยจะมีการให้คะแนนในเรื่องดังต่อไปนี้ 
\begin{enumerate}
    \item ความง่ายต่อการใช้งานของแอปพลิเคชัน 
    \item ความสะดวกในการใช้งานในตอนเริ่มต้นของแอปพลิเคชัน 
    \item ความดึงดูดในการใช้งานของแอปพลิเคชัน        
    \item ประโยชน์ที่มีของแอปพลิเคชัน 
\end{enumerate}
โดยที่ทั้ง 4 ข้อเป็นพิจราณาจากแนวคิดตาม The Four Elements of User Experience \cite{uxquantification} ที่ประกอบไป ด้วย 
\begin{enumerate}
    \item Usability ความใช้ง่ายในการใช้งาน เกี่ยวข้องกับสามารถในการใช้งาน รวมไปถึงความเหมาะสมการใช้งานกับผู้งานใช้ 
    \item Adaptability ความสามารถในงานปรับตัว กล่าวถึงระดับความยากง่ายของการใช้งานตั้งแต่จุดเริ่มต้น จนถึงจุดสิ้นสุดของระบบ โดยที่ผู้งานสามารถใช้งานได้อย่างคล่องแคล่ว 
    \item Desirability ความพึงพอใจ คือเมื่อใช้งานแล้วผู้ได้รับประสบการณ์ที่ดีในจากใช้งานของระบบ 
    \item Value คุณค่าของระบบ คือระบบที่ผู้ใช้เข้ามาใช้งานมีความสอดคล้องกับความต้องการของผู้ใช้ 
\end{enumerate}